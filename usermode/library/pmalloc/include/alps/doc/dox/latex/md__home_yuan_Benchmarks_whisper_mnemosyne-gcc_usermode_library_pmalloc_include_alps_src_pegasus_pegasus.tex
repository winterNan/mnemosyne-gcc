\hypertarget{md__home_yuan_Benchmarks_whisper_mnemosyne-gcc_usermode_library_pmalloc_include_alps_src_pegasus_pegasus_pegasuspage}{}\section{P\+E\+G\+A\+S\+U\+S\+: Persistent Global Address Space for Universal Sharing       }\label{md__home_yuan_Benchmarks_whisper_mnemosyne-gcc_usermode_library_pmalloc_include_alps_src_pegasus_pegasus_pegasuspage}
A P\+Ersistent Global Address Space for Universal Sharing (P\+E\+G\+A\+S\+US) object (\hyperlink{classalps_1_1AddressSpace}{alps\+::\+Address\+Space}) provides a shared address space between multiple worker processes. The address space comprises multiple persistent memory regions, which are contiguous segments of the address space that are backed by fabric-\/attached memory (F\+AM). Each process has its own P\+E\+G\+A\+S\+US object instance that it can use to map and access shared persistent memory regions. For emulation purposes, we also provide an implementation of persistent memory regions on top of an in-\/memory file system (T\+M\+P\+FS).

Persistent regions (\hyperlink{classalps_1_1Region}{alps\+::\+Region}) are instantiated by binding and mapping region files (\hyperlink{classalps_1_1RegionFile}{alps\+::\+Region\+File}) into the address space. As each process may memory map the region file at a different location, each region type provides smart pointers (\hyperlink{group__SMARTPOINTERS}{Smart Pointers}) for referencing locations within the region in a manner that is independent of mapping address. 